
\begin{figure*}
\centering
\tikzexternaldisable
\tikzumlset{font=\footnotesize}
\begin{tikzpicture}
%
%
\begin{umlpackage}[fill=none]{marietta}
       
               
        % CLASS ConstraintNormState
        \umlabstract[x=+8, y=3.5, fill=yellow!50]{ConstraintFunction}
        {
        }
        {
          \umlvirt{+constraint(x: \primdouble[~], u: \primdouble[~],}\\
          \quad \umlvirt{ w: \primdouble[~]):\primdouble[~] }
        }%
        \umlnote[x=2, y=3.5, width=2.8cm]{ConstraintFunction}
        {
	    Functions \(\phi_t(x,u,w)\)
	}
        \umlclass[x=+8,y=0, fill=yellow!50]{ConstraintNormState}
        {
	  -c: \primdouble
        }
        {
          +ConstraintNormState(c: \primdouble)\\
          {+constraint(x: \primdouble[~], u: \primdouble[~],}\\
          \quad { w: \primdouble[~]): \primdouble[~] }
        }%
        \umlnote[x=2, y=0, width=2.8cm]{ConstraintNormState}
        {
	    \(\phi_t(x,u,w)=\|x\|^2 - c\)
	}        
\end{umlpackage}
        
        
\umlimpl[arg1=impl]{ConstraintNormState}{ConstraintFunction}
\end{tikzpicture}

\tikzexternalenable
\caption{UML of constraint functions. These are functions of the form \(\phi_t:\R^{n_x}\times \R^{n_u}\times \R^{n_w}\ni (x,u,w) \mapsto \phi_t(x,u,w) \in \R\). We need to impose constraints of the form \(r_t[\phi_t(x,u,w)]\leq 0\) as explained in the paper. An example of such a function is \texttt{marietta.ConstraintNormState} which corresponds to \(\phi_t(x,u,w) = \|x\|^2 - c\), for some given \(c>0\).}
\end{figure*}